\documentclass[11pt,a4paper]{article}

% --------------------
% Paquetes básicos
% --------------------
\usepackage[utf8]{inputenc}
\usepackage[T1]{fontenc}
\usepackage{lmodern}
\usepackage[spanish]{babel}

\usepackage{amsmath, amssymb}
\usepackage{graphicx}
\usepackage{booktabs}
\usepackage{hyperref}
\usepackage{cite}

% Márgenes
\usepackage{geometry}
\geometry{margin=2.5cm}

% Espaciado
\usepackage{setspace}
\onehalfspacing

\title{\textbf{Resumen del Paper:}\\
\large Curiosity-Driven Exploration by Self-supervised Prediction}

\author{Ignacio Monardes}

\date{\today}

\begin{document}

\maketitle

\section*{Referencia}
\noindent
\textbf{Autores:} Ying et al. \\
\textbf{Título:} \textit{Assessing Adaptive World Models in Machines with Novel Games} \\
\textbf{Conferencia/Journal:} ICML 2017 \\
\textbf{Link:} \url{https://arxiv.org/pdf/2507.12821}

\section{Motivación}

Explica:
\begin{itemize}
    
    \item ¿Qué problema aborda el paper?
    Los seres humanos somos muy buenos jugando videojuegos porque entendemos los World Models mediante un world model induction. 
    Mientras jugamos aprendemos las lógicas de los mundos de forma muy rápida. Los algoritmos no son tan buenos aprendiendo los world models.    

    Se quiere medir como los agentes de WMs son capaces de ``inducir'' estos world models. 
    Para esto, se crearon unos \textit{novel games}, de tal forma que se pueda medir que tan fácil se adaptan
    los algoritmos a nuevos ambientes y a nuevas reglas/interacciones las cuales pueda hipotetizar y validar rápidamente.

    Se sugiere que es insuficiente predecir la imagen/estado futuro para entender las mecanicas de los juegos. Los
    mundos son dinámicos y cambian de forma constante. Por contraste, los humanos hacemos un \textit{sample efficiency}


    \item ¿Por qué es importante?
    
    Esta métrica es importante ya que en la realidad vivimos en un mundo que cambia constantemente, donde 
    hay nuevas reglas, nuevos climas, nuevos pisos, nuevas interacciones, etc. Por eso es importante
    que si algún día queramos que los agentes interactuen en el mundo real sean capaces de poder adaptarse
    a nuevos ambientes de forma veloz.



    \item ¿Qué limitaciones tienen los enfoques previos?
    ¨
    Las métricas anteriores principalmente se basaban en instancias no vistas. Ahora con estos nuevos
    ambientes se busca que el agente aprenda reglas/mecanismos que no son conocidos en un inicio. 
    Para que el agente obtenga buenos resultados se requiere que el agente logre inferir dinámicas latentes
    y estructuras causales. 

\end{itemize}

\section{Idea Principal}
Describe la contribución central del paper en alto nivel:
\begin{itemize}
    \item ¿Qué proponen?

    Primero, proponen que hay dos tipos de world models, los instanciados que son para instancias particulares
    y los abstractos los cuales son más abstractos. Un ejemplo de un instanciado es un mapa de nueva york. 
    Un ejemplo de un WM abstracto es un 

    \item ¿Qué lo hace diferente a trabajos anteriores?
\end{itemize}

% --------------------
\section{Metodología}
% --------------------
Explica brevemente:
\begin{itemize}
    \item Modelo / algoritmo principal


    \item Componentes clave
    
    
    \item Supuestos importantes


\end{itemize}

Si es necesario, puedes incluir ecuaciones:
\begin{equation}
    J(\theta) = \mathbb{E}_{\pi_\theta} \left[ \sum_t \gamma^t r_t \right]
\end{equation}

\section{Experimentos}
Describe:
\begin{itemize}
    \item Entornos o datasets

    \item Métricas usadas
    
    \item Baselines

\end{itemize}

\section{Resultados}
Resume los resultados más importantes:
\begin{itemize}
    \item ¿Supera a los baselines?

    \item ¿En qué escenarios funciona mejor / peor?

\end{itemize}

Opcional: referencia a figuras/tablas del paper.

\section{Discusión Crítica}
\begin{itemize}
    \item Fortalezas

    
    \item Debilidades
    
    
    \item Supuestos cuestionables
    
    
    \item Qué no queda claro


\end{itemize}

\section{Conclusiones}

\section*{Comentario Personal}
Tu opinión:
\begin{itemize}
    \item ¿Te parece una buena contribución?

    \item ¿La usarías en tu investigación?

\end{itemize}

\bibliographystyle{plain}
\bibliography{references}

\end{document}
