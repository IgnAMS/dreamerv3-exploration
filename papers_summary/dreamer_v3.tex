\documentclass[11pt,a4paper]{article}

% --------------------
% Paquetes básicos
% --------------------
\usepackage[utf8]{inputenc}
\usepackage[T1]{fontenc}
\usepackage{lmodern}
\usepackage[spanish]{babel}

\usepackage{amsmath, amssymb}
\usepackage{graphicx}
\usepackage{booktabs}
\usepackage{hyperref}
\usepackage{cite}



% Márgenes
\usepackage{geometry}


\renewcommand{\thesubsection}{\arabic{subsection}}
\setcounter{secnumdepth}{1} % Comentar para que aparezcan los números en los subsections 

\geometry{margin=2.5cm}

% Espaciado
\usepackage{setspace}
\onehalfspacing

\title{\textbf{Resumen del Paper:}\\
\large }

\author{Ignacio Monardes}

\date{\today}

\begin{document}

\maketitle

\section*{Referencia}
\noindent
\textbf{Autores: Danijar Hafner, et al.}  \\
\textbf{Título:} \textit{Mastering Diverse Domains trough World Models} \\
\textbf{Conferencia/Journal: Nature}   \\
\textbf{Link:} \url{https://arxiv.org/pdf/2301.04104}

\section{Motivación}
\subsection{Problema presente} 
 


\subsection{Relevancia del problema}

\subsection{Agujero en la literatura / limitaciones previas}


\section{Idea Principal}
\subsection{Idea concisa}
   
\subsection{Innovación}

\section{Metodología}

\subsection{Modelamiento} 


\subsection{Assumptions}
    

\subsection{Arquitectura}


\section{Experimentos}
\subsection{Entornos o datasets}
    

\subsection{Baselines}


\subsection{Settings}


\subsection{Nuevos escenarios}


\section{Resultados}


\subsection{¿En qué escenarios funciona mejor / peor?}

\section{Discusión Crítica}
\subsection{Fortalezas}

\subsection{Debilidades} 

\subsection{Supuestos cuestionables} 

\subsection{Qué no queda claro} 

\section{Conclusiones}

\section*{Comentario Personal}
\subsection{¿Es una buena contribución?}

\subsection{¿Lo usaría en mi investigación?}


\bibliographystyle{plain}
\bibliography{references}

\end{document}
