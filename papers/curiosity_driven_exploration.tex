\documentclass[11pt,a4paper]{article}

% --------------------
% Paquetes básicos
% --------------------
\usepackage[utf8]{inputenc}
\usepackage[T1]{fontenc}
\usepackage{lmodern}
\usepackage[spanish]{babel}

\usepackage{amsmath, amssymb}
\usepackage{graphicx}
\usepackage{booktabs}
\usepackage{hyperref}
\usepackage{cite}

% Márgenes
\usepackage{geometry}
\geometry{margin=2.5cm}

% Espaciado
\usepackage{setspace}
\onehalfspacing

\title{\textbf{Resumen del Paper:}\\
\large Curiosity-Driven Exploration by Self-supervised Prediction}

\author{Ignacio Monardes}

\date{\today}

\begin{document}

\maketitle

\section*{Referencia}
\noindent
\textbf{Autores:} Deepak Pathak et al. \\
\textbf{Título:} \textit{Curiosity-Driven Exploration by Self-supervised Prediction} \\
\textbf{Conferencia/Journal:} ICML 2017 \\
\textbf{Link:} \url{https://arxiv.org/abs/1705.05363}

% --------------------
\section{Motivación}
% --------------------
Explica:
\begin{itemize}
    \item ¿Qué problema aborda el paper?
    En muchos ambientes las rewards son sparse, por lo que una manera de incentivar al agente a que explore es dar una reward intrinseca por explorar.
    Se define curiosidad como 


    \item ¿Por qué es importante?

    \item ¿Qué limitaciones tienen los enfoques previos?

\end{itemize}

% --------------------
\section{Idea Principal}
% --------------------
Describe la contribución central del paper en alto nivel:
\begin{itemize}
    \item ¿Qué proponen?
    \item ¿Qué lo hace diferente a trabajos anteriores?
\end{itemize}

% --------------------
\section{Metodología}
% --------------------
Explica brevemente:
\begin{itemize}
    \item Modelo / algoritmo principal
    \item Componentes clave
    \item Supuestos importantes
\end{itemize}

Si es necesario, puedes incluir ecuaciones:
\begin{equation}
    J(\theta) = \mathbb{E}_{\pi_\theta} \left[ \sum_t \gamma^t r_t \right]
\end{equation}

% --------------------
\section{Experimentos}
% --------------------
Describe:
\begin{itemize}
    \item Entornos o datasets
    \item Métricas usadas
    \item Baselines
\end{itemize}

% --------------------
\section{Resultados}
% --------------------
Resume los resultados más importantes:
\begin{itemize}
    \item ¿Supera a los baselines?
    \item ¿En qué escenarios funciona mejor / peor?
\end{itemize}

Opcional: referencia a figuras/tablas del paper.

% --------------------
\section{Discusión Crítica}
% --------------------
Tu análisis:
\begin{itemize}
    \item Fortalezas
    \item Debilidades
    \item Supuestos cuestionables
    \item Qué no queda claro
\end{itemize}

% --------------------
\section{Conclusiones}
% --------------------

% --------------------
\section*{Comentario Personal (opcional)}
% --------------------
Tu opinión:
\begin{itemize}
    \item ¿Te parece una buena contribución?
    \item ¿La usarías en tu investigación?
\end{itemize}

% --------------------
\bibliographystyle{plain}
\bibliography{references}

\end{document}
